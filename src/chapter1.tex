\chapter{第一章 Aa 1}

章序号采用阿拉伯数字,章序号与标题名之间空一个汉字符。采用黑体三号字,居中书写,单倍行距,段前空24磅,段后空18磅。论文的摘要、目录、主要符号对照表、参考文献、附录、致谢、学位论文独创性声明及学位论文知识产权权属声明、个人简历、在学期间发表的学术论文与研究成果等部分的标题与章标题属于同一等级,也使用上述格式;英文摘要部分的标题“Abstract”采用Arial体三号字加粗。

在写作指南中并没有给出章节目录英文字体格式,本模板按下面的组合定义格式:

\begin{enumerate}
	\item 宋体 + Times New Roman
	\item 黑体 + Arial
	\item 公式字体选择为:Latin Modern Math
	\item 编程字体选择为:黑体 + Courier New
\end{enumerate}

\section{一级节标题 Aa 1}

一级节标题,例如:“2.1  实验装置与实验方法”。
节标题序号与标题名之间空一个汉字符(下同)。采用黑体四号(14pt)字居左书写,行距为固定值20磅,段前空24磅,段后空6磅。


\subsection{二级节标题}

二级节标题,例如:“2.1.1  实验装置”。
采用黑体13pt字居左书写,行距为固定值20磅,段前空12磅,段后空6磅。


\subsubsection{三级节标题}

三级节标题,例如:“2.1.2.1  归纳法”。
采用黑体小四号(12pt)字居左书写,行距为固定值20磅,段前空12磅,段后空6磅。一般情况下不建议使用三级节标题。


\section{论文段落的文字部分(正文)}

采用小四号(12pt)字,汉字用宋体,英文用Times New Roman体,两端对齐书写,首行缩进2个字符。行距为固定值20磅(段落中有数学表达式时,可根据表达需要设置该段的行距),段前空0磅,段后空0磅。
论文的附录、致谢、学位论文独创性声明及学位论文知识产权权属声明部分部分中的正文也使用上述格式。

\subsection{脚注}

正文中某句话需要具体注释、且注释内容与正文内容关系不大时可以采用脚注方式。在正文中需要注释的句子结尾处用“圆圈数字”样式的数字编排序号,以“上标”字体标示在需要注释的句子末尾。在当页下部书写脚注内容。
脚注内容采用宋体小五号字,按两端对齐格式书写,单倍行距,段前段后均为0。脚注的序号按页编排,不同页的脚注要重新编号。详细规定见本页脚注。 

在这里设置了,一页脚注个数不能超过9个。

脚注测试~\footnote{每页脚注从1开始,下一页编号清零,重新计数。},见页脚\footnote{每页脚注从1开始,下一页编号清零,重新计数。每页脚注从1开始,下一页编号清零,重新计数。每页脚注从1开始,下一页编号清零,重新计数。每页脚注从1开始,下一页编号清零,重新计数。}。

\subsection{参考文献}

新疆大学研究生学位论文中的参考文献采用GB/T 7714—2015推荐的顺序编码制格式著录。参考文献(即引文出处)的类型以单字母方式标识。

“参考文献”四个字的格式与章标题的格式相同。参考文献表的正文部分用五号字,汉字用宋体,英文用Times New Roman体,两端对齐,行距采用固定值16磅,段前空3磅,段后空0磅,缩进左右侧均为0。
参考文献的内容要尽量写在同一页内。遇有被迫分页的情况,可通过“留白”或微调本页行距的方式尽量将同一条文献内容放在一页。
关于参考文献的著录格式以及在正文中的标注方法详细见第3章。

参考文献引用测试~\cite{born1981}。

参考文献引用测试~\cite{knuth1984}。

多个参考文献引用测试~\cite{born1981,knuth1984}。


\subsection{超链接}

\url{https://cn.bing.com/}