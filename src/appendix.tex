\chapter{附录名称}

\section{附录要求}

附录是与论文内容密切相关、但编入正文又影响整篇论文编排的条理和逻辑性的一些资料,例如某些重要的数据表格、数学推导、计算程序、统计表、计算机打印输出件等,是论文主体的补充内容,可根据需要设置。

附录的格式与正文相同,并依顺序用大写字母A,B,C,……编序号,如附录A,附录B,附录C,……。只有一个附录时也要编序号,即附录A。每个附录应有标题。附录序号与附录标题之间空一个汉字符。例如:“附录A  2011年新疆设施农业天敌昆虫统计数据”。

附录中的图、表、数学表达式、参考文献等另行编序号,与正文分开,一律用阿拉伯数字编码,但在数码前冠以附录的序号,例如“图A.1”,“表B.2”,“式(C-3)”等。


\chapter{附录2}

\section{附录内容}

公式

\begin{equation}
	y=kx+b
\end{equation}

\begin{theorem}
	定理
\end{theorem}

表格

\begin{table}[htb]
	\centering
	\caption{表格}
	\begin{tabular}{llllll}
		\toprule[1.5pt]
		& A & B & C & D & E \\
		\midrule[1pt]
		1 & 000 & 111 & Aa & - & 表格	\\
		2 & 000 & 111 & Aa & - & 表格	\\
		3 & 000 & 111 & Aa & - & 表格	\\
		\bottomrule[1.5pt]
	\end{tabular}
\end{table}

图

\begin{figure}[hbt]
	\centering
	\includegraphics[width=0.4\textwidth]{xju_logo}
	\caption{图片名称}
\end{figure}